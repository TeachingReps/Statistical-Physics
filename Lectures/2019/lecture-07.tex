% !TEX spellcheck = en_US
% !TEX spellcheck = LaTeX
\documentclass[letterpaper,english,10pt]{article}
\usepackage{%
	amsfonts,%
	amsmath,%	
	amssymb,%
	amsthm,%
	babel,%
	bbm,%
	%biblatex,%
	caption,%
	centernot,%
	color,%
	enumerate,%
	%enumitem,%
	epsfig,%
	epstopdf,%
	etex,%
	fancybox,%
	framed,%
	fullpage,%
	%geometry,%
	graphicx,%
	hyperref,%
	latexsym,%
	mathptmx,%
	mathtools,%
	multicol,%
	pgf,%
	pgfplots,%
	pgfplotstable,%
	pgfpages,%
	proof,%
	psfrag,%
	%subfigure,%	
	tikz,%
	times,%
	ulem,%
	url,%
	xcolor,%
	mathpazo
}

\definecolor{shadecolor}{gray}{.95}%{rgb}{1,0,0}
\usepackage[margin=1in,top=0.75in]{geometry}
\usepackage[mathscr]{eucal}
\usepgflibrary{shapes}
\usepgfplotslibrary{fillbetween}
\usetikzlibrary{%
  arrows,%
  backgrounds,%
  chains,%
  decorations.pathmorphing,% /pgf/decoration/random steps | erste Graphik
  decorations.text,% 
  matrix,%
  positioning,% wg. " of "
  fit,%
  patterns,%
  petri,%
  plotmarks,%
  scopes,%
  shadows,%
  shapes.misc,% wg. rounded rectangle
  shapes.arrows,%
  shapes.callouts,%
  shapes%
}

%\pgfplotsset{compat=newest} %<------ Here
\pgfplotsset{compat=1.11} %<------ Or use this one

\theoremstyle{plain}
\newtheorem{thm}{Theorem}[section]
\newtheorem{lem}[thm]{Lemma}
\newtheorem{prop}[thm]{Proposition}
\newtheorem{cor}[thm]{Corollary}
\newtheorem{clm}[thm]{Claim}

\theoremstyle{definition}
\newtheorem{defn}[thm]{Definition}
\newtheorem{conj}[thm]{Conjecture}
\newtheorem{exmp}[thm]{Example}
\newtheorem{assum}[thm]{Assumptions}

\theoremstyle{remark}
\newtheorem{rem}{Remark}
\newtheorem{note}{Note}

\newcommand{\Cov}{\operatorname{Cov}}
%\newcommand{\det}{\operatorname{det}}
\newcommand{\Real}{\mathbb{R}}
\newcommand{\tr}{\operatorname{tr}}
\newcommand{\Var}{\operatorname{Var}}

%\renewcommand{\proof}[1]{\begin{proof}#1\end{proof}}
\newcommand{\EQ}[1]{\begin{equation*}#1\end{equation*}}
\newcommand{\EQN}[1]{\begin{equation}#1\end{equation}}
\newcommand{\eq}[1]{\begin{align*}#1\end{align*}}
\newcommand{\meq}[2]{\begin{xalignat*}{#1}#2\end{xalignat*}}
\newcommand{\norm}[1]{\left\lVert#1\right\rVert}
\newcommand{\abs}[1]{\left\lvert#1\right\rvert}
\newcommand{\expect}[1]{\mathbb{E}\left[{#1}\right]}
\newcommand{\prob}[1]{\mathbb{P}\left[{#1}\right]}
\newcommand{\given}{\; \big\vert \;} 
\newcommand{\set}[1]{\left\{#1\right\}} 
\newcommand{\indicator}[1]{1_{\set{#1}}} 
\newcommand{\red}[1]{\textcolor{red}{#1}} 

\newcommand{\D}{\mathbb{D}}
\newcommand{\E}{\mathbb{E}}
\newcommand{\N}{\mathbb{N}}
\renewcommand{\P}{\mathbb{P}}
\newcommand{\Q}{\mathbb{Q}}
\newcommand{\R}{\mathbb{R}}
\newcommand{\Z}{\mathbb{Z}}

\newcommand{\bU}{\mathbf{1}}
\newcommand{\bx}{\mathbf{x}}

\newcommand{\cB}{\mathcal{B}}
\newcommand{\cC}{\mathcal{C}}
\newcommand{\cF}{\mathcal{F}}
\newcommand{\cG}{\mathcal{G}}
\newcommand{\cT}{\mathcal{T}}
\newcommand{\cX}{\mathcal{X}}

\newcommand{\sB}{\mathscr{B}}
\newcommand{\sC}{\mathscr{C}}
\newcommand{\sE}{\mathscr{E}}
\newcommand{\sF}{\mathscr{F}}
\newcommand{\sG}{\mathscr{G}}
\newcommand{\sH}{\mathscr{H}}
\newcommand{\sL}{\mathscr{L}}
\newcommand{\sS}{\mathscr{S}}
\newcommand{\sT}{\mathscr{T}}
\newcommand{\sX}{\mathscr{X}}


% Debug
\newcommand{\todo}[1]{\begin{color}{blue}{{\bf~[TODO:~#1]}}\end{color}}

% a few handy macros

\renewcommand{\le}{\leqslant}
\renewcommand{\ge}{\geqslant}
\newcommand\matlab{{\sc matlab}}
\newcommand{\goto}{\rightarrow}
\newcommand{\bigo}{{\mathcal O}}
%\newcommand{\half}{\frac{1}{2}}
%\newcommand\implies{\quad\Longrightarrow\quad}
\newcommand\reals{{{\rm l} \kern -.15em {\rm R} }}
\newcommand\complex{{\raisebox{.043ex}{\rule{0.07em}{1.56ex}} \hskip -.35em {\rm C}}}


% macros for matrices/vectors:

% matrix environment for vectors or matrices where elements are centered
\newenvironment{mat}{\left[\begin{array}{ccccccccccccccc}}{\end{array}\right]}
\newcommand\bcm{\begin{mat}}
\newcommand\ecm{\end{mat}}

% matrix environment for vectors or matrices where elements are right justifvied
\newenvironment{rmat}{\left[\begin{array}{rrrrrrrrrrrrr}}{\end{array}\right]}
\newcommand\brm{\begin{rmat}}
\newcommand\erm{\end{rmat}}

% for left brace and a set of choices
%\newenvironment{choices}{\left\{ \begin{array}{ll}}{\end{array}\right.}
\newcommand\when{&\text{if~}}
\newcommand\otherwise{&\text{otherwise}}
% sample usage:
%  \delta_{ij} = \begin{choices} 1 \when i=j, \\ 0 \otherwise \end{choices}


% for labeling and referencing equations:
\newcommand{\eql}{\begin{equation}\label}
\newcommand{\eqn}[1]{(\ref{#1})}
% can then do
%  \eql{eqnlabel}
%  ...
%  \end{equation}
% and refer to it as equation \eqn{eqnlabel}.  


% some useful macros for finite difference methods:
\newcommand\unp{U^{n+1}}
\newcommand\unm{U^{n-1}}

% for chemical reactions:
\newcommand{\react}[1]{\stackrel{K_{#1}}{\rightarrow}}
\newcommand{\reactb}[2]{\stackrel{K_{#1}}{~\stackrel{\rightleftharpoons}
   {\scriptstyle K_{#2}}}~}


\makeatletter
\def\th@plain{%
  \thm@notefont{}% same as heading font
  \itshape % body font
}
\def\th@definition{%
  \thm@notefont{}% same as heading font
  \normalfont % body font
}
\makeatother
\date{}

%opening
\title{Lecture-07: The Thermodynamic Limit}
%\author{Ajay Kumar Badita}

\begin{document}
\maketitle
\section{The thermodynamic limit}
The main purpose of statistical physics is to understand the macroscopic behaviour of a large number, $N \gg1$, of microscopic components (atoms, molecules, etc.) under simple local interactions. 
For example, in the case of water in a bottle, the number of $H_2 O$ molecules $N$ is typically of order $10^{23}$ (18g of water contains approximately $6 \times10^{23}$ molecules), and this huge number leads physicists to focus on the $N \to \infty$ limit, also called \textbf{the thermodynamic limit}.

\subsection{The intensive thermodynamic potentials}
For large $N$, the thermodynamic potentials are proportional to $N$. 
The \textbf{intensive thermodynamic potentials} $f (\beta)$, $u (\beta)$, $s (\beta)$ are defined as follows.

\begin{defn}[Intensive thermodynamic potentials]
Denoting the thermodynamic potentials for $N$ particle system as $F_N (\beta), U_N (\beta), S_N (\beta)$ for the free energy, the internal energy, and the canonical entropy respectively. 
We can define the \textbf{free energy density},  the \textbf{energy density}, and the \textbf{entropy density} as 
\meqn{3}{
\label{eqn:density}
&f(\beta) = \lim_{N \to \infty} \frac{F_N (\beta)}{N}, &&
u(\beta) = \lim_{N \to \infty} \frac{U_N (\beta)}{N}, &&
s(\beta) = \lim_{N \to \infty} \frac{S_N (\beta)}{N}.
}
\end{defn}

Partition function $Z (\beta)$ is a sum of exponentials, and hence is smooth and analytic. 
It follows that the free energy $F_N (\beta) = -\frac{1}{\beta}\ln Z(\beta)$ is also analytic. 
\begin{defn}[Phase transition]
We say that a \textbf{phase transition} occurs, whenever the free energy density $f (\beta)$ is non-analytic. 
\end{defn}
Since the free entropy density $\phi (\beta) = -\beta f(\beta)$ is convex, the free energy density is necessarily continuous whenever it exists.
The phase transitions correspond to qualitative changes in the underlying physical system.
\begin{defn}[Types of singularities]
Often, the non-analyticities occur at isolated points say $\beta_c$.
\begin{itemize}
\item The free energy density is continuous, but its derivative with respect to $\beta$ is discontinuous at $\beta_c$. 
This singularity is called a \textbf{first-order phase transition}.
\item The free energy and its first derivative are continuous, but the second derivative is discontinuous at $\beta_c$. 
This is called a \textbf{second-order phase transition}.
\end{itemize}
\end{defn}


\subsection{Energy spectrum and Micro-canonincal entropy density}
When the number of particles $N$ grows, the volume of the configuration space increases exponentially, 
i.e. $|\sX_N| = |\sX|^N$. 
We have seen before that the system is likely to be found in lowest-energy configurations with high probability at low temperatures. 
From the definition of Boltzmann distribution, it is easy to check that conditioned on the system to be at  certain energy level, it is equally likely to be in any configuration with equal energy. 
Therefore, one of the important factor of interest is the number of configurations for any given energy level.  
This information is given by the \textbf{energy spectrum} of the system, 
defined by the set of states with energy in the interval $[E, E+\Delta)$, 
\EQ{
\Omega_\Delta (E) \triangleq \{ x \in \sX^N : E \le E(x) < E+\Delta\}.
}
The number of states in $\Omega_\Delta(E)$ is given by $\cN_\Delta (E) = \abs{\Omega_\Delta (E)}$. 
%where $\Omega_\Delta (E)$ represents the set of particles for which the energy lies in $E$ and $E+\Delta$ and $N_\Delta (E)$ gives the total number of such particles. 

The energy spectrum diverges exponentially in many systems as $N \to \infty$, if the energy is scaled linearly with $N$. 
More precisely, there exists a function $s(e)$ such that, given two numbers $e$ and $\delta > 0$, 
\EQN{
\label{eqn:MicroCanonicalEntropy}
s(e) = \lim_{N \to \infty} \frac{1}{N} \log \cN_{N\delta}(Ne) = \sup_{e'\in [e,e+\delta]} s(e').
}
Compactly, we can write this as 
\eq{
\cN_\Delta(E) \stackrel{.}{=}_N \exp\left[N s\left(\frac{E}{N}\right)\right], 
}
where the notation $\stackrel{.}{=}_N$ denotes that two quantities $A_N$ and $B_N$ are equal \textbf{to leading exponential order}, meaning $\lim_{N \to \infty} \frac{1}{N} \log (A_N/B_N)=0$.

The micro-canonical entropy density $s(e)$ conveys a great amount of information about the system, and is directly related to the intensive thermodynamic
potentials through a fundamental relation.

\begin{rem}
\label{rem:partitionFunctionWithEnergySpectrum}
Given the whole energy band divided into $\Delta$ intervals, the partition function can be now written as 

\eq{
Z_N (\beta) &= \sum_{x \in \mathcal{X_N}} \exp(-\beta E(x)) = \sum_{x \in \mathcal{X_N}} \sum_{k=-\infty}^{\infty} \indicator{k\Delta \le \frac{E(x)}{N} < (k+1)\Delta} \exp(-\beta N k\Delta)\\
& = \sum_{k=-\infty}^{\infty} \exp(-\beta N k\Delta) \sum_{x \in \mathcal{X_N}} \indicator{k\Delta \le \frac{E(x)}{N} < (k+1)\Delta} 
\stackrel{.}{=}_N \sum_{k=-\infty}^{\infty} \exp(-\beta N k\Delta) \exp(N s(k\Delta)\\
&= \sum_{k=-\infty}^{\infty} \exp[N (s(k\Delta)  -\beta k\Delta)]\\
&= \int \exp[N (s(e)-\beta e)] de
}
\end{rem}


\begin{prop}
If the micro-canonical entropy density~\eqref{eqn:MicroCanonicalEntropy} exists for any $e$ and if the limit in equation~\eqref{eqn:MicroCanonicalEntropy} is uniform in $e$, then the free entropy density ~\eqref{eqn:density} exists and is given by
\EQN{
\phi(\beta) = \max_e [s(e) -\beta e]
}
If the maximum of $s(e)-\beta e$ is unique, then the internal-energy density equals $arg\max[s(e)-\beta e]$.
\end{prop}

\begin{proof}
From the definition, the free entropy density $\phi(\beta)$ can be written as
\eq{
\phi(\beta) =  \lim_{N \to \infty} \frac{1}{N} \log(Z_N(\beta))
}
By using the remark ~\ref{rem:partitionFunctionWithEnergySpectrum},
\eq{
\phi(\beta) &=  \lim_{N \to \infty} \frac{1}{N} \sum_{k=-\infty}^{\infty} \exp[N (s(k\Delta)  -\beta k\Delta)]\\
&= \sup_k [s(k\Delta)  -\beta k\Delta]\\
&= \max_e [s(e)  -\beta e]
}
\end{proof}

\begin{shaded*} 
\begin{exmp}[$N$ identical two-level systems]
Let us $N$ identical two-level systems, i.e. $\sX _N = \sX \times \sX  \times \ldots \times \sX$, with $\sX = \{1,2\}$. We take the energy to be the sum of the single-system energies: $E(x) = E_{single}(x_1) + E_{single}(x_2) + \ldots + E_{single}(x_N) $ with $x_i \in \sX$. We set $E_{single}(1) = \epsilon_1, E_{single}(2) = \epsilon_2 > \epsilon_1$ and, $\Delta = \epsilon_2 - \epsilon_1$.

The energy spectrum of this model is quite simple. For any energy $E = N \epsilon + n\Delta$,there are $\binom{N}{n}$ configurations $x$ with $E(x) = E$. 
Where $n$ is the number of particles in level-2.

w.k.t 
\eq{
N_\Delta(E) = \binom{N}{n} \approx 2^{N \mathcal{H}(\frac{n}{N})} \approx 2^{N \mathcal{H} \big(\frac{E-N\epsilon_1}{N\Delta} \big)}
}
Therefore, using the definition ~\eqref{eqn:MicrocanonicalEntropy}, we get
\eq{
s(e) = \mathcal{H}\Big(\frac{e-\epsilon_1}{\Delta}\Big)\\
}
The free energy can be now computed as,
\eq{
f(\beta) &= -\frac{1}{\beta} \phi(\beta) = -\frac{1}{\beta} \sup_e [s(e)  -\beta e] = -\frac{1}{\beta} \sup_e [\mathcal{H}\Big(\frac{e-\epsilon_1}{\Delta}\Big)  -\beta e]\\
&= \epsilon_1 - \frac{1}{\beta}  log (1+\exp(-\beta \Delta))
}

The above equation can be obtained by first solving $ \sup_e [\mathcal{H}\Big(\frac{e-\epsilon_1}{\Delta}\Big) -\beta e]$ as by taking the first differention and set to zero, we get the saddle point as $ e=\frac{\epsilon_1 + \epsilon_2 \exp(-\beta \Delta)}{1+\exp(-\beta \Delta)}$. Substittuting this back and considering the fact that as $N \to \infty$ the expression $\exp(-\beta \Delta) \approx 1$ gives the value of $\sup_e$ as $-\beta\epsilon_1+log (1+\exp(-\beta \Delta))$.
\end{exmp}
\end{shaded*}


\section{Ferromagnets and Ising models}
Magnetic materials contain molecules with a magnetic moments, that tend to align the external magnetic field felt by the molecule and these magnetic moments of two different molecules interact with each other.In many materials, the energy is lower when the moments align. 

A simple mathematical model for considering a bunch of interacting moments is the Ising model, which describes the magnetic moments by Ising spins localized at the vertices of a certain region of a d-dimensional cubic lattice. Consider a region $\mathbb{L}$, a cube of side $L:\mathbb{L}=\{1,2, \dots, L\}^d$. At each coordinate $i \in \mathbb{L}$, the state of a particle is an Ising spin $\sigma_i \in \{-1,1\}$.

%\red{Insert figure here}

The system configuration $\underline{\sigma} = (\sigma_1, \sigma_2, \dots, \sigma_N)$ is given by assigning the values of all spins. The space of configuration is $\sX_N = \{+1,-1\}^L$, with $\sX= \{+1,-1\}$ and $N=L^d$.
The energy $E(\underline{\sigma})$ of a given configuration $\underline{\sigma}$ is given by,
\eq{
E(\underline{\sigma}) = - \sum_{(i,j)} \sigma_i \sigma_j - B \sum_{i \in \mathbb{L}} \sigma_i
}
where the sum over $(i,j)$ runs over all the unordered pairs of sites $i,j \in \mathbb{L}$ which are nearest neighbours and B is the applied external magnetic field.
Determining the free energy density $f(\beta)$ in the thermodynamic limit for this model is a non-trivial task. In 1924, Ernst Ising solved the $d=1$ case and showed the absence of phase transitions. In 1948, Lars Onsager solved the $d=2$ case, exhibiting the first soluble ‘finite-dimensional’ model with a second-order phase transition. In higher dimensions,the problem is unsolved, although many important features of the solution are well understood.

The two limiting cases that can be considered are, at high temperature, $\beta=0$, the energy no longer matters and the Boltzmann distribution is uniform over all configurations $\sigma$ as
\eq{
\mu_\beta (\sigma) = \frac{1}{2^N}
}
At low temperature, $\beta \to \infty$, the Boltzmann distribution concentrates onto the ground state(s). In the absence of external magnetic fiels, i.e $B=0$, the two degenerate ground states are given by,
\eq{
\underline{\sigma}^+ = (\sigma_i = +1, i \in [N])\\
\underline{\sigma}^- = (\sigma_i = -1, i \in [N])\\
}
If the magnetic field is set to some non-zero value, one of the two configuration dominates: $\underline{\sigma}^+$ if $B>0$ and $\underline{\sigma}^-$ if $B<0$.
\end{document} 