\documentclass[letterpaper,english,12pt]{article}
\usepackage{%
	amsfonts,%
	amsmath,%	
	amssymb,%
	amsthm,%
	babel,%
	bbm,%
	%biblatex,%
	caption,%
	centernot,%
	color,%
	enumerate,%
	%enumitem,%
	epsfig,%
	epstopdf,%
	etex,%
	fancybox,%
	framed,%
	fullpage,%
	%geometry,%
	graphicx,%
	hyperref,%
	latexsym,%
	mathptmx,%
	mathtools,%
	multicol,%
	pgf,%
	pgfplots,%
	pgfplotstable,%
	pgfpages,%
	proof,%
	psfrag,%
	%subfigure,%	
	tikz,%
	times,%
	ulem,%
	url,%
	xcolor,%
	mathpazo
}

\definecolor{shadecolor}{gray}{.95}%{rgb}{1,0,0}
\usepackage[margin=1in,top=0.75in]{geometry}
\usepackage[mathscr]{eucal}
\usepgflibrary{shapes}
\usepgfplotslibrary{fillbetween}
\usetikzlibrary{%
  arrows,%
  backgrounds,%
  chains,%
  decorations.pathmorphing,% /pgf/decoration/random steps | erste Graphik
  decorations.text,% 
  matrix,%
  positioning,% wg. " of "
  fit,%
  patterns,%
  petri,%
  plotmarks,%
  scopes,%
  shadows,%
  shapes.misc,% wg. rounded rectangle
  shapes.arrows,%
  shapes.callouts,%
  shapes%
}

%\pgfplotsset{compat=newest} %<------ Here
\pgfplotsset{compat=1.11} %<------ Or use this one

\theoremstyle{plain}
\newtheorem{thm}{Theorem}[section]
\newtheorem{lem}[thm]{Lemma}
\newtheorem{prop}[thm]{Proposition}
\newtheorem{cor}[thm]{Corollary}
\newtheorem{clm}[thm]{Claim}

\theoremstyle{definition}
\newtheorem{defn}[thm]{Definition}
\newtheorem{conj}[thm]{Conjecture}
\newtheorem{exmp}[thm]{Example}
\newtheorem{assum}[thm]{Assumptions}

\theoremstyle{remark}
\newtheorem{rem}{Remark}
\newtheorem{note}{Note}

\newcommand{\Cov}{\operatorname{Cov}}
%\newcommand{\det}{\operatorname{det}}
\newcommand{\Real}{\mathbb{R}}
\newcommand{\tr}{\operatorname{tr}}
\newcommand{\Var}{\operatorname{Var}}

%\renewcommand{\proof}[1]{\begin{proof}#1\end{proof}}
\newcommand{\EQ}[1]{\begin{equation*}#1\end{equation*}}
\newcommand{\EQN}[1]{\begin{equation}#1\end{equation}}
\newcommand{\eq}[1]{\begin{align*}#1\end{align*}}
\newcommand{\meq}[2]{\begin{xalignat*}{#1}#2\end{xalignat*}}
\newcommand{\norm}[1]{\left\lVert#1\right\rVert}
\newcommand{\abs}[1]{\left\lvert#1\right\rvert}
\newcommand{\expect}[1]{\mathbb{E}\left[{#1}\right]}
\newcommand{\prob}[1]{\mathbb{P}\left[{#1}\right]}
\newcommand{\given}{\; \big\vert \;} 
\newcommand{\set}[1]{\left\{#1\right\}} 
\newcommand{\indicator}[1]{1_{\set{#1}}} 
\newcommand{\red}[1]{\textcolor{red}{#1}} 

\newcommand{\D}{\mathbb{D}}
\newcommand{\E}{\mathbb{E}}
\newcommand{\N}{\mathbb{N}}
\renewcommand{\P}{\mathbb{P}}
\newcommand{\Q}{\mathbb{Q}}
\newcommand{\R}{\mathbb{R}}
\newcommand{\Z}{\mathbb{Z}}

\newcommand{\bU}{\mathbf{1}}
\newcommand{\bx}{\mathbf{x}}

\newcommand{\cB}{\mathcal{B}}
\newcommand{\cC}{\mathcal{C}}
\newcommand{\cF}{\mathcal{F}}
\newcommand{\cG}{\mathcal{G}}
\newcommand{\cT}{\mathcal{T}}
\newcommand{\cX}{\mathcal{X}}

\newcommand{\sB}{\mathscr{B}}
\newcommand{\sC}{\mathscr{C}}
\newcommand{\sE}{\mathscr{E}}
\newcommand{\sF}{\mathscr{F}}
\newcommand{\sG}{\mathscr{G}}
\newcommand{\sH}{\mathscr{H}}
\newcommand{\sL}{\mathscr{L}}
\newcommand{\sS}{\mathscr{S}}
\newcommand{\sT}{\mathscr{T}}
\newcommand{\sX}{\mathscr{X}}


% Debug
\newcommand{\todo}[1]{\begin{color}{blue}{{\bf~[TODO:~#1]}}\end{color}}

% a few handy macros

\renewcommand{\le}{\leqslant}
\renewcommand{\ge}{\geqslant}
\newcommand\matlab{{\sc matlab}}
\newcommand{\goto}{\rightarrow}
\newcommand{\bigo}{{\mathcal O}}
%\newcommand{\half}{\frac{1}{2}}
%\newcommand\implies{\quad\Longrightarrow\quad}
\newcommand\reals{{{\rm l} \kern -.15em {\rm R} }}
\newcommand\complex{{\raisebox{.043ex}{\rule{0.07em}{1.56ex}} \hskip -.35em {\rm C}}}


% macros for matrices/vectors:

% matrix environment for vectors or matrices where elements are centered
\newenvironment{mat}{\left[\begin{array}{ccccccccccccccc}}{\end{array}\right]}
\newcommand\bcm{\begin{mat}}
\newcommand\ecm{\end{mat}}

% matrix environment for vectors or matrices where elements are right justifvied
\newenvironment{rmat}{\left[\begin{array}{rrrrrrrrrrrrr}}{\end{array}\right]}
\newcommand\brm{\begin{rmat}}
\newcommand\erm{\end{rmat}}

% for left brace and a set of choices
%\newenvironment{choices}{\left\{ \begin{array}{ll}}{\end{array}\right.}
\newcommand\when{&\text{if~}}
\newcommand\otherwise{&\text{otherwise}}
% sample usage:
%  \delta_{ij} = \begin{choices} 1 \when i=j, \\ 0 \otherwise \end{choices}


% for labeling and referencing equations:
\newcommand{\eql}{\begin{equation}\label}
\newcommand{\eqn}[1]{(\ref{#1})}
% can then do
%  \eql{eqnlabel}
%  ...
%  \end{equation}
% and refer to it as equation \eqn{eqnlabel}.  


% some useful macros for finite difference methods:
\newcommand\unp{U^{n+1}}
\newcommand\unm{U^{n-1}}

% for chemical reactions:
\newcommand{\react}[1]{\stackrel{K_{#1}}{\rightarrow}}
\newcommand{\reactb}[2]{\stackrel{K_{#1}}{~\stackrel{\rightleftharpoons}
   {\scriptstyle K_{#2}}}~}


\makeatletter
\def\th@plain{%
  \thm@notefont{}% same as heading font
  \itshape % body font
}
\def\th@definition{%
  \thm@notefont{}% same as heading font
  \normalfont % body font
}
\makeatother
\date{}

%opening
\title{Lecture 7: The Thermodynamic Limit}
\author{Ajay Kumar Badita}

\begin{document}
\maketitle
\section{The Thermodynamic Limit}
The main purpose of statistical physics is to understand the macroscopic behaviour of a large number, $N\gg1$, of simple components (atoms, molecules, etc.) when they are brought together.
For example, in the case of water in a bottle, the number of $H_2 O$ molecules $N$ is typically of order $10^{23}$ (18g of water contains approximately $6x10^{23}$ molecules), and this huge number leads physicists to focus on the $N \rightarrow \infty$ limit, also called \textbf{the thermodynamic limit}.

\subsection{The intensive thermodynamic potentials}
For large N, the thermodynamic potentials are proportionl to $N$. The \textbf{intensive thermodynamic potentials} $f (\beta)$, $u (\beta)$, $s (\beta)$ are defined as follows.

\begin{defn}
The \textbf{free energy density} is defined by 
\eq{
f(\beta) = \lim_{N \rightarrow \infty} \frac{F_N (\beta)}{N}
}
\end{defn}

\begin{defn}
The \textbf{energy density} is defined by 
\eq{
u(\beta) = \lim_{N \rightarrow \infty} \frac{U_N (\beta)}{N}
}
\end{defn}

\begin{defn}
The \textbf{entropy density} is defined by 
\eq{
s(\beta) = \lim_{N \rightarrow \infty} \frac{S_N (\beta)}{N}
}
\end{defn}

where $F_N (\beta)$, $U_N (\beta)$, $S_N (\beta)$ the free energy, internal energy, and canonical entropy, respectively for a system with $N$  particle system.

\begin{defn}
$Z (\beta)$ is a function involving sum of exponentials, is smooth and analytic, there by the free energy $F_N (\beta)$ is also analytic.
Whenever the free energy density $f (\beta)$ is non-analytic, we say that a \textbf{phase transition} occurs.
Since the free entropy density $\phi (\beta) = -\beta f(\beta)$ is convex, the free energy density is necessarily continuous whenever it exits.
The phase transitions correspond to qualitative changes in the underlying physical system.
\end{defn}.  

\paragraph{Types of singularities:}
The non-analyticities occur at isolated points (say $\beta_c$).
\begin{itemize}
\item
	The free energy density is continuous, but its derivative with respect to $\beta$ is discontinuous at $\beta_c$. This singularity is called a \textbf{first-order phase transition}.
\item
	The free energy and its first derivative are continuous, but the second derivative is discontinuous at $\beta_c$. This is called a \textbf{second-order phase transition}.
\end{itemize}

\subsection{Energy spectrum and Micro canonincal entropy density}
When N grows, the volume of the configuration space increases exponentially:$|X_N| = |X|^N$ and most of the states are in lowest-energy configuration with high probability. Hence the important factor of interst is the number of configurations at a given energy. This information is given by the \textbf{energy spectrum} of the system,
\eq{
N_\triangle (E) = |\Omega_\triangle (E)|
}
\eq{
\Omega_\triangle (E) = \{ x \in \mathcal{X}_N : E \le E(x) < E+\triangle\}
}

where $\Omega_\triangle (E)$ represents the set of particles for which the energy lies in $E$ and $E+\triangle$ and $N_\triangle (E)$ gives the total number of such particles.
The energy spectrum diverges exponentially in many systems as $N \rightarrow \infty$, if the energy is scaled linealry with $N$. More precisely, there exists a function $s(e)$ such that, given two numbers $e$ and $\delta > 0$,

\eq{
s(e) = \lim_{N \rightarrow \infty} \frac{1}{N} \log N_{N\delta}(Ne) = \sup_{e'\in [e,e+\delta]} s(e')
}
In compact we can write this as,
\eq{
N_\triangle(E) \red{=} exp[N s\Big{(}\frac{E}{N}\Big{)}]
}

Where the notation \red{=} denotes that two quantities $A_N$ and $B_N$ are equal \textbf{to leading exponential order}, meaning $lim_{N \rightarrow \infty} \frac{1}{N} \log (A_N/B_N)=0$.

The microcanonical entropy density $s(e)$ conveys a great amount of information about the system, and is directly related to the intensive thermodynamic
potentials through a fundamental relation.

\begin{prop}

\end{prop}


\end{document} 