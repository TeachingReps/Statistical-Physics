% !TEX spellcheck = en_US
% !TEX spellcheck = LaTeX
\documentclass[letterpaper,english,10pt]{article}
\usepackage{%
	amsfonts,%
	amsmath,%	
	amssymb,%
	amsthm,%
	babel,%
	bbm,%
	%biblatex,%
	caption,%
	centernot,%
	color,%
	enumerate,%
	%enumitem,%
	epsfig,%
	epstopdf,%
	etex,%
	fancybox,%
	framed,%
	fullpage,%
	%geometry,%
	graphicx,%
	hyperref,%
	latexsym,%
	mathptmx,%
	mathtools,%
	multicol,%
	pgf,%
	pgfplots,%
	pgfplotstable,%
	pgfpages,%
	proof,%
	psfrag,%
	%subfigure,%	
	tikz,%
	times,%
	ulem,%
	url,%
	xcolor,%
	mathpazo
}

\definecolor{shadecolor}{gray}{.95}%{rgb}{1,0,0}
\usepackage[margin=1in,top=0.75in]{geometry}
\usepackage[mathscr]{eucal}
\usepgflibrary{shapes}
\usepgfplotslibrary{fillbetween}
\usetikzlibrary{%
  arrows,%
  backgrounds,%
  chains,%
  decorations.pathmorphing,% /pgf/decoration/random steps | erste Graphik
  decorations.text,% 
  matrix,%
  positioning,% wg. " of "
  fit,%
  patterns,%
  petri,%
  plotmarks,%
  scopes,%
  shadows,%
  shapes.misc,% wg. rounded rectangle
  shapes.arrows,%
  shapes.callouts,%
  shapes%
}

%\pgfplotsset{compat=newest} %<------ Here
\pgfplotsset{compat=1.11} %<------ Or use this one

\theoremstyle{plain}
\newtheorem{thm}{Theorem}[section]
\newtheorem{lem}[thm]{Lemma}
\newtheorem{prop}[thm]{Proposition}
\newtheorem{cor}[thm]{Corollary}
\newtheorem{clm}[thm]{Claim}

\theoremstyle{definition}
\newtheorem{axiom}[thm]{Axiom}
\newtheorem{defn}[thm]{Definition}
\newtheorem{conj}[thm]{Conjecture}
\newtheorem{exmp}[thm]{Example}
\newtheorem{exerc}[thm]{Exercise}
\newtheorem{assum}[thm]{Assumptions}

\theoremstyle{remark}
\newtheorem{rem}[thm]{Remark}
\newtheorem{note}[thm]{Note}

\newcommand{\Cov}{\operatorname{Cov}}
%\newcommand{\det}{\operatorname{det}}
\newcommand{\Real}{\mathbb{R}}
\newcommand{\tr}{\operatorname{tr}}
%\newcommand{\Var}{\operatorname{Var}}

\DeclareMathOperator{\sign}{sign}
%\renewcommand{\proof}[1]{\begin{proof}#1\end{proof}}
\newcommand{\EQ}[1]{\begin{equation*}#1\end{equation*}}
\newcommand{\EQN}[1]{\begin{equation}#1\end{equation}}
\newcommand{\eq}[1]{\begin{align*}#1\end{align*}}
\newcommand{\meq}[2]{\begin{xalignat*}{#1}#2\end{xalignat*}}
\newcommand{\norm}[1]{\left\lVert#1\right\rVert}
\newcommand{\abs}[1]{\left\lvert#1\right\rvert}
\newcommand{\expect}[1]{\mathbb{E}\left[{#1}\right]}
\newcommand{\prob}[1]{\mathbb{P}\left[{#1}\right]}
\newcommand{\given}{\; \big\vert \;} 
\newcommand{\set}[1]{\left\{#1\right\}} 
\newcommand{\indicator}[1]{\mathbb{1}_{\set{#1}}} 
\newcommand{\inner}[1]{\left\langle#1\right\rangle}
\newcommand{\red}[1]{\textcolor{red}{#1}} 
\newcommand{\E}[1]{\mathbb{E}\left[#1\right]}
\newcommand{\Var}[1]{\operatorname{Var}\left[#1\right]}

\newcommand{\D}{\mathbb{D}}
%\newcommand{\E}{\mathbb{E}}
\newcommand{\N}{\mathbb{N}}
\renewcommand{\P}{\mathbb{P}}
\newcommand{\Q}{\mathbb{Q}}
\newcommand{\R}{\mathbb{R}}
\newcommand{\Z}{\mathbb{Z}}

\newcommand{\bU}{\mathbf{1}}
\newcommand{\bx}{\mathbf{x}}

\newcommand{\cB}{\mathcal{B}}
\newcommand{\cC}{\mathcal{C}}
\newcommand{\cD}{\mathcal{D}}
\newcommand{\cF}{\mathcal{F}}
\newcommand{\cG}{\mathcal{G}}
\newcommand{\cH}{\mathcal{H}}
\newcommand{\cO}{\mathcal{O}}
\newcommand{\cT}{\mathcal{T}}
\newcommand{\cX}{\mathcal{X}}
\newcommand{\cY}{\mathcal{Y}}

\newcommand{\sA}{\mathscr{A}}
\newcommand{\sB}{\mathscr{B}}
\newcommand{\sC}{\mathscr{C}}
\newcommand{\sD}{\mathscr{D}}
\newcommand{\sE}{\mathscr{E}}
\newcommand{\sF}{\mathscr{F}}
\newcommand{\sG}{\mathscr{G}}
\newcommand{\sH}{\mathscr{H}}
\newcommand{\sL}{\mathscr{L}}
\newcommand{\dO}{\mathscr{O}}
\newcommand{\sS}{\mathscr{S}}
\newcommand{\sT}{\mathscr{T}}
\newcommand{\sX}{\mathscr{X}}
\newcommand{\sY}{\mathscr{Y}}
\newcommand{\sZ}{\mathscr{Z}}

% Debug
\newcommand{\todo}[1]{\begin{color}{blue}{{\bf~[TODO:~#1]}}\end{color}}

% a few handy macros

\renewcommand{\le}{\leqslant}
\renewcommand{\ge}{\geqslant}
\newcommand\matlab{{\sc matlab}}
\newcommand{\goto}{\rightarrow}
\newcommand{\bigo}{{\mathcal O}}
%\newcommand{\half}{\frac{1}{2}}
%\newcommand\implies{\quad\Longrightarrow\quad}
\newcommand\reals{{{\rm l} \kern -.15em {\rm R} }}
\newcommand\complex{{\raisebox{.043ex}{\rule{0.07em}{1.56ex}} \hskip -.35em {\rm C}}}


% macros for matrices/vectors:

% matrix environment for vectors or matrices where elements are centered
\newenvironment{mat}{\left[\begin{array}{ccccccccccccccc}}{\end{array}\right]}
\newcommand\bcm{\begin{mat}}
\newcommand\ecm{\end{mat}}

% matrix environment for vectors or matrices where elements are right justifvied
\newenvironment{rmat}{\left[\begin{array}{rrrrrrrrrrrrr}}{\end{array}\right]}
\newcommand\brm{\begin{rmat}}
\newcommand\erm{\end{rmat}}

% for left brace and a set of choices
%\newenvironment{choices}{\left\{ \begin{array}{ll}}{\end{array}\right.}
\newcommand\when{&\text{if~}}
\newcommand\otherwise{&\text{otherwise}}
% sample usage:
%  \delta_{ij} = \begin{choices} 1 \when i=j, \\ 0 \otherwise \end{choices}


% for labeling and referencing equations:
\newcommand{\eql}{\begin{equation}\label}
\newcommand{\eqn}[1]{(\ref{#1})}
% can then do
%  \eql{eqnlabel}
%  ...
%  \end{equation}
% and refer to it as equation \eqn{eqnlabel}.  


% some useful macros for finite difference methods:
\newcommand\unp{U^{n+1}}
\newcommand\unm{U^{n-1}}

% for chemical reactions:
\newcommand{\react}[1]{\stackrel{K_{#1}}{\rightarrow}}
\newcommand{\reactb}[2]{\stackrel{K_{#1}}{~\stackrel{\rightleftharpoons}
   {\scriptstyle K_{#2}}}~}


\makeatletter
\def\th@plain{%
  \thm@notefont{}% same as heading font
  \itshape % body font
}
\def\th@definition{%
  \thm@notefont{}% same as heading font
  \normalfont % body font
}
\makeatother
\date{}


\title{Lecture-18: Random Energy Model}


\begin{document}
\maketitle


\section{Condensation Phenomenon}
In the low temperature phase, a sub-exponential number of configurations dominate the Boltzmann measure. We call this phenomenon a \textit{condensation} of the measure onto this set of configurations.
In order to measure the degree of concentration of the measure, we define the \textbf{participation ratio function}
\EQ{
Y_N(\beta) \triangleq \sum_{j=1}^{2^N}\mu_\beta(j)^2 = \frac{\sum_{j=1}^{2^N}e^{-2\beta E_j}}{\left(\sum_{j=1}^{2^N}e^{-\beta E_j}\right)^2} = \frac{Z_N(2\beta)}{Z_N(\beta)^2}. 
}
In general, in the high temperature phase i.e. $\beta \leq \beta_c$, the partition function $Z_N(\beta) \doteq exp[N(\ln2 + \beta^2/4)]$. Taking the infinite temperature limit $(\beta \to 0)$, we have $\lim_{\beta \to 0}Y_N(\beta) = \frac{1}{2^{N}}$. This indicates that the participation is distributed evenly over all the $2^N$ levels,
hence we get $Y_N (0) = 2^{-N}$.\\ 
Similarly if we consider the low-temperature setup used in Chapter 2, we would get instead $\lim_{\beta \to \infty}Y_N(\beta) = \frac{1}{\abs{\sX_0}}$. 
In other words, very few states meaningfully contribute to the partition function.
\begin{itemize}
\item For the high-temperature phase of the REM, i.e., for $\beta \le \beta_c = 2\sqrt{\ln2}$, we have
\EQ{
\E Y_N(\beta)\indicator{\beta \le \beta_c} \stackrel{\cdot}{=} \frac{e^{N\phi(2\beta)}}{e^{2N\phi(\beta)}} = \frac{e^{N(\ln2 + \beta^2)}}{e^{2N(\ln 2+ \frac{\beta^2}{4})}} = e^{-N(\ln2 - \frac{\beta^2}{2})}, 
}
which decays exponentially with $N$ for $\beta $. 
The measure is therefore broadly distributed at high temperatures, i.e., $\lim_{N \to \infty}\E Y_N(\beta)\indicator{\beta  < \frac{\beta_c}{\sqrt{2}}} = 0$. 
 
 \item At low temperatures, a naive calculation gives
 \EQ{
 \E Y_N(\beta)\indicator{\beta > \beta_c} = 1,
 }
which is inaccurate -- although a single (ground) state is indeed in control when $\beta \to \infty$. 
The problem is due to the fact that the sub-exponential terms in $N$ should be taken into account in the calculation. 
Because they have been neglected in the calculation of $Z_N(\beta)$, they do not appear here. 
In reality, $Y_N (\beta)$ is finite and fluctuates a lot from sample to sample (as suggested by the above calculation). 
It is shown in Appendix~\ref{appendix:Condensation} that in the thermodynamic limit, we instead get
\EQ{
\E Y(\beta) = \begin{cases}
0, & \beta < \beta_c\\
1- \frac{\beta_c}{\beta}, & \beta \ge \beta_c.
\end{cases}
}
Hence, we see that condensation begins at $\beta_c$, and that the measure becomes increasingly concentrated until the ground state dominates, as $T \to 0$.  
\end{itemize}

\section{Annealed vs. Quenched}
When the free energy density is self-averaging, the value of $\phi_N$ is roughly the same for all samples and can be calculated through the quenched average
\EQ{
\phi_{N,q} \triangleq \frac{1}{N}\E\ln Z_N, 
}
but in general it can be quite difficult to compute this quantity. 
The annealed average
\EQ{
\phi_{N,a} \triangleq \frac{1}{N}\ln \E Z_N, 
}
allows the energy levels themselves to thermalize and gives here
\EQ{
\E Z_N =\E\sum_{i=1}^{2^N}e^{-\beta E_i} =  2^N\E e^{-\beta E_i} = 2^N\frac{1}{\sqrt{\pi N}}\int_{E \in \R}e^{-\beta E-\frac{E^2}{N}} = e^{N(\ln2+\frac{\beta^2}{4})},
}
which means that $\phi_{N,a} = \frac{\beta^2}{4}+\ln2$. 
By chance, for this model the result is the same as the high-temperature result, but it completely misses the phase transition and gives a negative entropy $s(\beta) = \ln2-\frac{\beta^2}{4}$ for temperatures below $\frac{1}{\beta_c}$.  
This negative entropy is a clear signal of failure of the calculation. 
The reason for the failure is as follows. 
For a given sample with a free energy density $f_N (\beta)$, the partition function behaves as 
\EQ{
Z_N = \exp[-\beta N f_N(\beta)].
}
Self-averaging means that $f_N(\beta)$ has small sample-to-sample fluctuations. 
These fluctuations, however, do exist and their impact is amplified by the factor of $N$ in the exponent. The partition function can thus be dominated by some very rare samples, i.e., those with an anomalously low value of $f_N(\beta)$. 

Consider, for instance, the low-temperature limit. 
We know that in almost all samples the configuration with the lower energy density is found at $E_i \approx N\epsilon^\ast$. 
There exist, however, exceptional samples, where one configuration has an energy smaller $E_i = -N\epsilon$ with $\epsilon > \epsilon^\ast$. 
Because these samples are exponentially rare (with probability $\stackrel{\cdot}{=}2^Ne^{-N\epsilon^2}$, they are irrelevant for the quenched average, but they dominate the annealed average.
%\begin{appendices}
\appendix
\section{REM Condensation}
\label{appendix:Condensation}
Using the identity 
\EQ{
\frac{1}{Z^2} = \int_{0}^{\infty}te^{-tZ}dt 
}
Since the energies $\{E_i\}$ in the REM  are i.i.d random variables we get for $M = 2^N$,
\eq{
\E Y_N(\beta) &=\E \frac{Z_N(2\beta)}{Z_N(\beta)^2} = \E \int_{0}^\infty Z_N(2\beta)e^{-tZ_N(\beta)}dt\\
&= \E\int_{0}^\infty \left(\sum_{j=1}^Me^{-2\beta E_j}\right) t e^{-t\sum_{j=1}^Me^{-\beta E_j}} dt 
= M \E\int_{0}^\infty t e^{-2\beta E_1-t\sum_{j=1}^Me^{-\beta E_j}} dt\\
&= M \E\int_{0}^\infty t e^{-2\beta E_1-te^{-\beta E_1} -t\sum_{j=2}^Me^{-\beta E_j}} dt = M\int_0^\infty ta(t)[1-b(t)]^{M-1}dt, 
}
where we have separated the averaging over $E_1$ and $E_{i \neq1}$ as 
\meq{2}{
&a(t)\triangleq \E e^{-2\beta E-te^{-\beta E}}, &&b(t)\triangleq \E\left(1- e^{-te^{-\beta E}}\right). 
}
In order to get information about the scaling of $P(E)$ at low temperatures, 
we need to expand the function around the ground state energy. 
The strategy below is to identify its first sub-extensive correction. 
We thus consider a regime where for a system of $2^N$ states the probability of having a state of energy $-N\epsilon_0$ is high, hence we get 
\EQ{
2^NP_N(-N\epsilon_0) = \frac{ 2^N}{\sqrt{\pi N}}e^{-N\epsilon^2} \approx 1, 
}
which corresponds to an estimate of the ground state in a finite $N$ system. 
Taking the log on both sides and solving for $\epsilon_0$ gives in the large $N$ limit 
\EQ{
\epsilon_0  = \sqrt{\ln2 - \frac{1}{N}\ln\sqrt{\pi N}} 
= \sqrt{\ln2} - \frac{\ln\sqrt{\pi N}}{N2\sqrt{\ln2}}  + O(N^{-2}) 
\approx \epsilon^\ast - \frac{1}{2\epsilon^\ast}\frac{\ln\sqrt{\pi N}}{N},
}
 which includes the asymptotic correction to the ground state energy $\epsilon^\ast = \sqrt{\ln 2}$. 
 %Note the missing N in Eq. (5.22) ? see book errata.
Because we expect the energy in the low-temperature phase to be concentrated around its ground
state $\epsilon_0$, we make an expansion around it by change of variables 
\meq{2}{
&E = -N\epsilon_0 + u, &&t = ze^{-N\beta\epsilon_0}.
}
We then get that in the large $N$ limit 
\EQ{
\ln P(E)=-\ln\sqrt{\pi N} - \frac{(N\epsilon_0+u)^2}{N}
=-N\epsilon_\ast^2+2\epsilon_\ast u + \Theta(1/N) \approx \ln\left(2^{-N}e^{\beta_cu}\right),
}
where $\beta_c = 2\sqrt{\ln2}$, and hence we get that in the low energy limit 
\EQ{
P(E) \approx \frac{1}{M}e^{\beta_cu}.
}
We thus obtain (note that large $u$ corrections quickly vanish) 
\eq{
a(t) &\equiv \int exp[-2\beta E -te^{-\beta E}]dP(E)\\
&\simeq \frac{1}{M}e^{2N\beta\epsilon_0}\int_{-\infty}^\infty  e^{\beta_cu-2\beta u-ze^{-\beta u}} du = \frac{e^{2N\beta\epsilon_0}}{M\beta}z^{(\frac{\beta_c}{\beta}-2)} \Gamma(2-\frac{\beta_c}{\beta})\\
b(t) &\equiv \int [1 - exp[-te^{-\beta E}]dP(E)\\
&\simeq \frac{1}{M}\int_{-\infty}^{\infty}du e^{-\beta_cu}\left(1 -  e^{-ze^{-\beta u}}\right) 
= -\frac{1}{M\beta}z^{\frac{\beta_c}{\beta}}\Gamma\left(-\frac{\beta_c}{\beta}\right).
}
\begin{exerc}
$\int_{-\infty}^\infty  e^{(\beta_c-2\beta) u-ze^{-\beta u}} du = \frac{z^{(\frac{\beta_c}{\beta}-2)}}{\beta}\Gamma(2-\frac{\beta_c}{\beta})$\\
Using change of variables, let $e^{e^{-\beta u}} = p$ and $2 - \frac{\beta_c}{\beta}= r$. We then have $u = \frac{-\ln(\ln(p))}{\beta}$ and $du = \frac{- dp}{\beta p \ln p}$.\\
\eq{
\int_{-\infty}^\infty  e^{(\beta_c-2\beta) u-ze^{-\beta u}} du &= \int_{\infty}^0 \frac{(\ln p)^{(2 - \frac{\beta_c}{\beta})} p^{-z}}{-\beta p \ln p} dp = \frac{-1}{\beta} \int_{\infty}^0 (\ln p)^{r-1} p^{-(z+1)}dp
= \frac{1}{\beta} \int_0^{\infty} t^{r-1} e^{-zt} dt \text{ \Big[ setting }\ln p = t \Big]\\
&= \frac{1}{\beta} \int_0^{\infty} x^{r-1} \frac{e^{-x}}{z^r} dx \text{ \Big[change of variables with }zt= x \Big] = \frac{z^{(\frac{\beta_c}{\beta}-2)}}{\beta}\Gamma(2-\frac{\beta_c}{\beta})
}
\end{exerc}
\begin{exerc}
$\int_{-\infty}^{\infty} e^{-\beta_cu}\left(1 -  e^{-ze^{-\beta u}}\right) du 
= -\frac{1}{\beta}z^{\frac{\beta_c}{\beta}}\Gamma\left(-\frac{\beta_c}{\beta}\right)$\\
Using change of variables, let $e^{e^{-\beta u}} = p$ and $- \frac{\beta_c}{\beta}= r$. We then have $u = \frac{-\ln(\ln(p))}{\beta}$ and $du = \frac{- dp}{\beta p \ln p}$.\\
\eq{
\int_{-\infty}^{\infty} e^{-\beta_cu}\left(1 -  e^{-ze^{-\beta u}}\right) du &= \int_{\infty}^0 \frac{(\ln p)^r [1-p^{-z}]}{-\beta p \ln p} dp = \frac{1}{\beta} \int_0^{\infty} (\ln p)^{r-1} [1 -p^{-z}]dp/p = \frac{1}{\beta} \int_0^{\infty} t^{r-1} [1-e^{-zt}] dt \text{ \Big[ set }\ln p = t \Big]\\
&= \frac{z^{-r}}{\beta} \int_0^{\infty} x^{r-1} [1-e^{-x}] dx \text{ \Big[set }zt= x \Big] = \frac{z^{-r}}{\beta} \int_0^{\infty} x^{r-1} e^{-x} [e^x - 1] dx = - \frac{z^{-r}}{\beta}\Gamma(r)    
}
The last equality follows from integration by parts.\\
\end{exerc}
Noting that for large N, from the limit definition of exponential function $exp(x) = \lim_{n \to \infty}(1 + \frac{x}{n})^n$ the expression $[1 - b(t)]^{M-1} $ in $\E Y_N(\beta)$ can be approximated by $e^{-Mb(t)}$.\\
\\
Using the above approximation and after doing the proper change of variables, one obtains in the low temperature limit
\EQ{
\E Y(\beta) = M\int_0^\infty ta(t)e^{-Mb(t)}dt 
= \frac{1}{\beta}\Gamma(2-\frac{\beta_c}{\beta})\int_0^\infty z^{\frac{\beta_c}{\beta}-1}e^{\frac{1}{\beta}\Gamma(-\frac{\beta_c}{\beta})z^{\frac{\beta_c}{\beta}}} + \Theta(N^{-1}) = 1 - \frac{\beta_c}{\beta}, 
}
%in the large $N$ limit and low temperature limit. 
%\end{appendices}
\end{document}
