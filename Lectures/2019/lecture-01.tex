% !TEX spellcheck = en_US
% !TEX spellcheck = LaTeX
\documentclass[letterpaper,10pt,english]{article}
\usepackage{%
	amsfonts,%
	amsmath,%	
	amssymb,%
	amsthm,%
	babel,%
	bbm,%
	%biblatex,%
	caption,%
	centernot,%
	color,%
	enumerate,%
	%enumitem,%
	epsfig,%
	epstopdf,%
	etex,%
	fancybox,%
	framed,%
	fullpage,%
	%geometry,%
	graphicx,%
	hyperref,%
	latexsym,%
	mathptmx,%
	mathtools,%
	multicol,%
	pgf,%
	pgfplots,%
	pgfplotstable,%
	pgfpages,%
	proof,%
	psfrag,%
	%subfigure,%	
	tikz,%
	times,%
	ulem,%
	url,%
	xcolor,%
	mathpazo
}

\definecolor{shadecolor}{gray}{.95}%{rgb}{1,0,0}
\usepackage[margin=1in,top=0.75in]{geometry}
\usepackage[mathscr]{eucal}
\usepgflibrary{shapes}
\usepgfplotslibrary{fillbetween}
\usetikzlibrary{%
  arrows,%
  backgrounds,%
  chains,%
  decorations.pathmorphing,% /pgf/decoration/random steps | erste Graphik
  decorations.text,% 
  matrix,%
  positioning,% wg. " of "
  fit,%
  patterns,%
  petri,%
  plotmarks,%
  scopes,%
  shadows,%
  shapes.misc,% wg. rounded rectangle
  shapes.arrows,%
  shapes.callouts,%
  shapes%
}

%\pgfplotsset{compat=newest} %<------ Here
\pgfplotsset{compat=1.11} %<------ Or use this one

\theoremstyle{plain}
\newtheorem{thm}{Theorem}[section]
\newtheorem{lem}[thm]{Lemma}
\newtheorem{prop}[thm]{Proposition}
\newtheorem{cor}[thm]{Corollary}
\newtheorem{clm}[thm]{Claim}

\theoremstyle{definition}
\newtheorem{defn}[thm]{Definition}
\newtheorem{conj}[thm]{Conjecture}
\newtheorem{exmp}[thm]{Example}
\newtheorem{assum}[thm]{Assumptions}

\theoremstyle{remark}
\newtheorem{rem}{Remark}
\newtheorem{note}{Note}

\newcommand{\Cov}{\operatorname{Cov}}
%\newcommand{\det}{\operatorname{det}}
\newcommand{\Real}{\mathbb{R}}
\newcommand{\tr}{\operatorname{tr}}
\newcommand{\Var}{\operatorname{Var}}

%\renewcommand{\proof}[1]{\begin{proof}#1\end{proof}}
\newcommand{\EQ}[1]{\begin{equation*}#1\end{equation*}}
\newcommand{\EQN}[1]{\begin{equation}#1\end{equation}}
\newcommand{\eq}[1]{\begin{align*}#1\end{align*}}
\newcommand{\meq}[2]{\begin{xalignat*}{#1}#2\end{xalignat*}}
\newcommand{\norm}[1]{\left\lVert#1\right\rVert}
\newcommand{\abs}[1]{\left\lvert#1\right\rvert}
\newcommand{\expect}[1]{\mathbb{E}\left[{#1}\right]}
\newcommand{\prob}[1]{\mathbb{P}\left[{#1}\right]}
\newcommand{\given}{\; \big\vert \;} 
\newcommand{\set}[1]{\left\{#1\right\}} 
\newcommand{\indicator}[1]{1_{\set{#1}}} 
\newcommand{\red}[1]{\textcolor{red}{#1}} 

\newcommand{\D}{\mathbb{D}}
\newcommand{\E}{\mathbb{E}}
\newcommand{\N}{\mathbb{N}}
\renewcommand{\P}{\mathbb{P}}
\newcommand{\Q}{\mathbb{Q}}
\newcommand{\R}{\mathbb{R}}
\newcommand{\Z}{\mathbb{Z}}

\newcommand{\bU}{\mathbf{1}}
\newcommand{\bx}{\mathbf{x}}

\newcommand{\cB}{\mathcal{B}}
\newcommand{\cC}{\mathcal{C}}
\newcommand{\cF}{\mathcal{F}}
\newcommand{\cG}{\mathcal{G}}
\newcommand{\cT}{\mathcal{T}}
\newcommand{\cX}{\mathcal{X}}

\newcommand{\sB}{\mathscr{B}}
\newcommand{\sC}{\mathscr{C}}
\newcommand{\sE}{\mathscr{E}}
\newcommand{\sF}{\mathscr{F}}
\newcommand{\sG}{\mathscr{G}}
\newcommand{\sH}{\mathscr{H}}
\newcommand{\sL}{\mathscr{L}}
\newcommand{\sS}{\mathscr{S}}
\newcommand{\sT}{\mathscr{T}}
\newcommand{\sX}{\mathscr{X}}


% Debug
\newcommand{\todo}[1]{\begin{color}{blue}{{\bf~[TODO:~#1]}}\end{color}}

% a few handy macros

\renewcommand{\le}{\leqslant}
\renewcommand{\ge}{\geqslant}
\newcommand\matlab{{\sc matlab}}
\newcommand{\goto}{\rightarrow}
\newcommand{\bigo}{{\mathcal O}}
%\newcommand{\half}{\frac{1}{2}}
%\newcommand\implies{\quad\Longrightarrow\quad}
\newcommand\reals{{{\rm l} \kern -.15em {\rm R} }}
\newcommand\complex{{\raisebox{.043ex}{\rule{0.07em}{1.56ex}} \hskip -.35em {\rm C}}}


% macros for matrices/vectors:

% matrix environment for vectors or matrices where elements are centered
\newenvironment{mat}{\left[\begin{array}{ccccccccccccccc}}{\end{array}\right]}
\newcommand\bcm{\begin{mat}}
\newcommand\ecm{\end{mat}}

% matrix environment for vectors or matrices where elements are right justifvied
\newenvironment{rmat}{\left[\begin{array}{rrrrrrrrrrrrr}}{\end{array}\right]}
\newcommand\brm{\begin{rmat}}
\newcommand\erm{\end{rmat}}

% for left brace and a set of choices
%\newenvironment{choices}{\left\{ \begin{array}{ll}}{\end{array}\right.}
\newcommand\when{&\text{if~}}
\newcommand\otherwise{&\text{otherwise}}
% sample usage:
%  \delta_{ij} = \begin{choices} 1 \when i=j, \\ 0 \otherwise \end{choices}


% for labeling and referencing equations:
\newcommand{\eql}{\begin{equation}\label}
\newcommand{\eqn}[1]{(\ref{#1})}
% can then do
%  \eql{eqnlabel}
%  ...
%  \end{equation}
% and refer to it as equation \eqn{eqnlabel}.  


% some useful macros for finite difference methods:
\newcommand\unp{U^{n+1}}
\newcommand\unm{U^{n-1}}

% for chemical reactions:
\newcommand{\react}[1]{\stackrel{K_{#1}}{\rightarrow}}
\newcommand{\reactb}[2]{\stackrel{K_{#1}}{~\stackrel{\rightleftharpoons}
   {\scriptstyle K_{#2}}}~}


\makeatletter
\def\th@plain{%
  \thm@notefont{}% same as heading font
  \itshape % body font
}
\def\th@definition{%
  \thm@notefont{}% same as heading font
  \normalfont % body font
}
\makeatother
\date{}

\title{Lecture-01: Random Variables and Entropy}
\author{}

\begin{document}
\maketitle

\section{Random Variables} 

Our main focus will be on the behavior of large sets of discrete random variables. 
%From a mathematical perspective this class deals primarily with probability and, in particular, discrete random variables. 
\begin{defn} 
A \textbf{discrete random variable}, $X$, is defined by following information: 
(i) $\sX$ : the finite set of values that it may take, 
(ii) $p_X : \sX \to [0, 1]$: the probability it takes each value $x \in X$ . 
Of course, the probability distribution $p_X$ must satisfy the normalization condition $\sum_{x \in X}p_X (x) = 1$. 
If there is no ambiguity, we may use $p(x)$ to denote $p_X (x)$. 
\end{defn}
\begin{shaded*}\begin{exmp}
Let the random variable $X$ denote the sum of two fair 6-sided dice. 
Then, $\sX = \set{2, 3, \dots, 12}$ and 
\EQ{
p_X (x) = \frac{6 - \abs{7 - x}} {36}.
}
\end{exmp}\end{shaded*}
\begin{defn}
An \textbf{event} $A \subseteq \sX$ is a subset of values. 
The probability of an event is denoted 
\EQ{
\P(X \in A)=\P(A) = \sum_{x \in A}p_X(x) = \sum_{x \in A}\P(X =x).
}
Also, an event is sometimes defined in words, $A = ``X \text{ is even}''$.
\end{defn}
\begin{shaded*}\begin{exmp} 
If X is the sum of two fair 6-sided dice and $A = ``X \text{ is even}''$. Then, 
\EQ{
\P(X \text{ is even}) = \P(A) = \sum_{x \in A}p_X (x) = \frac{1 + 3 + 5 + 5 + 3 + 1}{36} = \frac{1}{2}.
}
\end{exmp}\end{shaded*}
\begin{defn} 
For a discrete random variable, the expected value (or average) of $f : \sX \to \R$ is denoted
\EQ{
\E f = \E{f(X)} = \sum_{x \in \sX}p_X(x)f(x).
}
Mathematically, $\E{}$ can be seen as a linear operator from the space of real functions on $\sX$ to the set of real numbers. 
Thus,
\EQ{
\E{af(X) + bg(X)} = a\E{f(X)} + b\E{g(X)}.
}
\end{defn}

\begin{shaded*}\begin{exmp}
If $X$ is the sum of two fair 6-sided dice and $f(x) = (x - 7)^2$, 
then 
\EQ{
\E{(X-7)^2} =\sum_{x \in A}p_X(x)(x-7)^2 = \frac{2(1\cdot 5^2 +2\cdot 4^2 +3\cdot 3^2 +4\cdot 2^2 +5\cdot 1^2)}{36} = \frac{105}{18}.
}
Since the mean is $\E{X} = 7$, this actually equals the variance of $X$.
\end{exmp}\end{shaded*}
\begin{defn} 
A continuous random variable, $X$, taking values on the set $\sX = \R^d$ or in some smooth finite-dimensional manifold is defined by its cumulative distribution function $\P(X \le x)$, where $X \le x$ is used to denote
$X_i \le x_i$ for $i = 1, \dots, d$. For such a r.v., the probability measure with respect to the
infinitesimal element $dx$ is denoted by $dp_X (x)$. 
For a measurable event $\sA \subseteq X$ , this gives 
\EQ{
\P(X \in \sA) = \int_{\sA}dp_X(x) = \int\indicator{x \in \sA}dp_X(x),
}
where the indicator function $\indicator{s}$ is $1$ if the logical statement $s$ is true and $0$ otherwise. 
If $p_X$ admits a density, with respect to Lebesgue measure, then it will be denoted by $p_X(x)$. 
In this case, we can write
\EQ{
\P(X \in \sA) = \int_{\sA}p_X(x)dx = \int\indicator{x \in \sA}p_X(x)dx.
}
\end{defn}
\begin{shaded*}\begin{exmp}
If $X$ is a continuous random variable defined, for $a, b \in \R$ with $a < b$, by
\EQ{
\P(X \le x) = 
\begin{cases}
0 &  x < a\\
\frac{x-a}{b-a} & a \le x \le b\\
1 & x > b,
\end{cases}
}
then it is uniform on $[a, b]$ and its density is given by $p_X (x) = \frac{1}{b-a}\indicator{x \in [a,b]}$.
\end{exmp}\end{shaded*}

\begin{defn} 
The expected value and variance of a function $f : \R^d \to \R$ of a continuous
random variable $X \in \R^d$ are given by
\eq{
&\E f = \E{f(X)} = \int f(x)dp_X(x),\\
&\Var{f} =\Var{f(X)}=\E{(f(X)- \E{f(X)})^2}=\E{f(X)^2} -\E{f(X)}^2.
}
\end{defn}
\begin{shaded*}\begin{exmp} 
If $X$ is a continuous random variable that is uniform on $[a, b]$, 
then its mean and variance are given by
\eq{
&\E{X} = \int_a^b \frac{x}{b-a}dx = \frac{b^2-a^2}{2(b-a)} = \frac{b+a}{2},\\
&\Var{X} =  \int_a^b \frac{x^2}{b-a}dx -\left(\frac{b+a}{2}\right)^2= \frac{b^3-a^3}{3(b-a)} -\left(\frac{b+a}{2}\right)^2= \frac{(b-a)^2}{12}.
}
\end{exmp}\end{shaded*}
     
\section{Entropy}
In statistical mechanics, the entropy is proportional to the logarithm of the number of resolvable microstates associated with a macrostate. 
In classical mechanics, this quantity contains an arbitrary additive constant associated with the size of a microstate that is considered resolvable. 
In quantum mechanics, there is a natural limit to resolvability and this constant is related to the Planck constant. 
For random variables, Shannon chose the following definition which is similar in spirit. 
\begin{defn} 
The \textbf{entropy} (in bits) of a discrete random variable $X$ with probability distribution $p(x)$ is denoted
\EQ{
H(X) \triangleq -\sum_{x \in \sX}p(x)\log_2p(x)=\E{\frac{1}{\log_2p(X)}},
}
where $0 \log_2 0 = 0$ by continuity. 
The notation $H (p)$ is used to denote $H (X )$ when $X \sim p(x)$. 
When there is no ambiguity, $H$ will be used instead of $H (X)$. 
The unit of entropy is determined by the base of the logarithm with base-2 resulting in ``bits'' and the natural log (i.e., base-e) resulting in ``nats''. 
\end{defn}
\begin{rem}
Roughly speaking, the entropy $H (X)$ measures the uncertainty in the random variable $X$.
\end{rem}
\begin{shaded*}\begin{exmp} 
If $X$ is uniform, then $p(x) = \frac{1}{\abs{\sX}}$ and 
\EQ{
H (X) = \E{\log_2 \frac{1}{\abs{\sX}}} = \log_2 \abs{\sX}.
}
Choosing $\abs{\sX} = 2$, we see that a uniform random bit has exactly $\log_2 2 = 1$ bit of entropy.
\end{exmp}\end{shaded*}
\begin{shaded*}\begin{exmp}
Let $X$ be a binary r.v. defined by $p(0) = 1-q$ and $p(1) = q$. 
In this case, we have
\EQ{
H(X)= \cH(q) = q\log_2 \frac{1}{q}+(1-q)\log_2\frac{1}{1-q}, 
}
where $\cH(q)$ is called the binary entropy function. 
This function is concave and symmetric about $q = \frac{1}{2}$. 
It also satisfies $\cH(0) = \cH(1) = 0$ and $\cH(1/2) = 1$.
\end{exmp}\end{shaded*}
\begin{shaded*}\begin{exmp}
The number of length-$n$ binary sequences with exactly $qn$ ones is given by $\binom{n}{qn}$. 
Using Stirling's formula, $n! = \sqrt{2\pi n}(\frac{n}{e})^n(1 + O(\frac{1}{n})$, we see that 
\eq{
\binom{n}{qn} &= \frac{n!}{(n-qn)!(qn)!}\\
&= \frac{\sqrt{2\pi n}(\frac{n}{e})^n(1+O(\frac{1}{n})}{\sqrt{2\pi n(1-q)}(\frac{n(1-q)}{e})^n(1+O(\frac{1}{n(1-q)})\sqrt{2\pi qn}(\frac{qn}{e})^n(1+O(\frac{1}{qn})}\\
&= \frac{1}{\sqrt{2\pi n q(1-q)}}2^{n\cH(q)}\left(1 + O\left(\frac{1}{nq(1-q)}\right)\right).
}
\end{exmp}\end{shaded*}
\begin{rem} 
This shows that the binary entropy determines the exponential growth rate of the number of binary sequences with a fixed fraction of ones. 
In fact, this is a fundamental property of the entropy. 
More generally, we will see that the entropy $H (X)$ is the exponential growth rate of the number of length-$n$ sequences (i.e., there are roughly $2^{nH(X)}$ such sequences) where the fraction of $x's$ converges to $np(x)$. 
This also implies that $nH (X)$ is essentially equal to the minimum number of binary digits required to index all length-$n$ sequences of this type. 
\end{rem}
\begin{lem} 
Basic properties of entropy:
\begin{enumerate}
\item (non-negativity) $H (X) \ge 0$ with equality iff $X$ is constant. 
\begin{proof} 
%For each $x \in \sX$, we have $\log_2\frac{1}{p(x)} \ge 0$ and, 
If $X$ is not constant, there is an $x_0 \in \sX$ with $p(x_0) \in (0,1)$. 
Thus, 
\EQ{
H (X) \ge p(x_0) \log_2(1/p(x_0)) \ge 0.
}
\end{proof}
\item (decomposition rule) For any partition $A = (A_1, A_2, \dots, A_m)$ of $\sX$, we have 
\EQ{
H(p) = H(p_A) + \sum_{i=1}^m p(A_i)H(p_i), 
}
where we define $p_A(i) = p(A_i) = \sum_{x \in A_i} p(x)$ for $i \in [m]$ and $p_i(x) = \frac{p(x)}{p(A_i)}$ for $x \in A_i$.
\begin{proof}
Observe that
\EQ{
H(X) = \sum_{i=1}^m\sum_{x \in A_i}p(x)\log_2\frac{1}{p(x)} = H(p_A) + \sum_{i=1}^mp(A_i)\sum_{x \in A_i}\frac{p(x)}{p(A_i)}\log_2\frac{p(A_i)}{p(x)} 
}
\end{proof}
\end{enumerate}
\end{lem}
\begin{shaded*}\begin{exmp} 
Compute the entropy of the distribution $p(x) = \begin{bmatrix}0.125  & 0.375 & 0.25 &0.25\end{bmatrix}$. Using decomposition with $A_1 = \set{1, 2}$ and $A_2 = \set{3, 4}$, we get 
\EQ{
H (p) = 1 + 0.5\cH(1/4) + 0.5 \approx 1.9056.
}
\end{exmp}\end{shaded*}
\end{document}
