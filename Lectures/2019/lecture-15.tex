% !TEX spellcheck = en_US
% !TEX spellcheck = LaTeX
\documentclass[letterpaper,english,10pt]{article}
\usepackage{%
	amsfonts,%
	amsmath,%	
	amssymb,%
	amsthm,%
	babel,%
	bbm,%
	%biblatex,%
	caption,%
	centernot,%
	color,%
	enumerate,%
	%enumitem,%
	epsfig,%
	epstopdf,%
	etex,%
	fancybox,%
	framed,%
	fullpage,%
	%geometry,%
	graphicx,%
	hyperref,%
	latexsym,%
	mathptmx,%
	mathtools,%
	multicol,%
	pgf,%
	pgfplots,%
	pgfplotstable,%
	pgfpages,%
	proof,%
	psfrag,%
	%subfigure,%	
	tikz,%
	times,%
	ulem,%
	url,%
	xcolor,%
	mathpazo
}

\definecolor{shadecolor}{gray}{.95}%{rgb}{1,0,0}
\usepackage[margin=1in,top=0.75in]{geometry}
\usepackage[mathscr]{eucal}
\usepgflibrary{shapes}
\usepgfplotslibrary{fillbetween}
\usetikzlibrary{%
  arrows,%
  backgrounds,%
  chains,%
  decorations.pathmorphing,% /pgf/decoration/random steps | erste Graphik
  decorations.text,% 
  matrix,%
  positioning,% wg. " of "
  fit,%
  patterns,%
  petri,%
  plotmarks,%
  scopes,%
  shadows,%
  shapes.misc,% wg. rounded rectangle
  shapes.arrows,%
  shapes.callouts,%
  shapes%
}

%\pgfplotsset{compat=newest} %<------ Here
\pgfplotsset{compat=1.11} %<------ Or use this one

\theoremstyle{plain}
\newtheorem{thm}{Theorem}[section]
\newtheorem{lem}[thm]{Lemma}
\newtheorem{prop}[thm]{Proposition}
\newtheorem{cor}[thm]{Corollary}
\newtheorem{clm}[thm]{Claim}

\theoremstyle{definition}
\newtheorem{axiom}[thm]{Axiom}
\newtheorem{defn}[thm]{Definition}
\newtheorem{conj}[thm]{Conjecture}
\newtheorem{exmp}[thm]{Example}
\newtheorem{exerc}[thm]{Exercise}
\newtheorem{assum}[thm]{Assumptions}

\theoremstyle{remark}
\newtheorem{rem}[thm]{Remark}
\newtheorem{note}[thm]{Note}

\newcommand{\Cov}{\operatorname{Cov}}
%\newcommand{\det}{\operatorname{det}}
\newcommand{\Real}{\mathbb{R}}
\newcommand{\tr}{\operatorname{tr}}
%\newcommand{\Var}{\operatorname{Var}}

\DeclareMathOperator{\sign}{sign}
%\renewcommand{\proof}[1]{\begin{proof}#1\end{proof}}
\newcommand{\EQ}[1]{\begin{equation*}#1\end{equation*}}
\newcommand{\EQN}[1]{\begin{equation}#1\end{equation}}
\newcommand{\eq}[1]{\begin{align*}#1\end{align*}}
\newcommand{\meq}[2]{\begin{xalignat*}{#1}#2\end{xalignat*}}
\newcommand{\norm}[1]{\left\lVert#1\right\rVert}
\newcommand{\abs}[1]{\left\lvert#1\right\rvert}
\newcommand{\expect}[1]{\mathbb{E}\left[{#1}\right]}
\newcommand{\prob}[1]{\mathbb{P}\left[{#1}\right]}
\newcommand{\given}{\; \big\vert \;} 
\newcommand{\set}[1]{\left\{#1\right\}} 
\newcommand{\indicator}[1]{\mathbb{1}_{\set{#1}}} 
\newcommand{\inner}[1]{\left\langle#1\right\rangle}
\newcommand{\red}[1]{\textcolor{red}{#1}} 
\newcommand{\E}[1]{\mathbb{E}\left[#1\right]}
\newcommand{\Var}[1]{\operatorname{Var}\left[#1\right]}

\newcommand{\D}{\mathbb{D}}
%\newcommand{\E}{\mathbb{E}}
\newcommand{\N}{\mathbb{N}}
\renewcommand{\P}{\mathbb{P}}
\newcommand{\Q}{\mathbb{Q}}
\newcommand{\R}{\mathbb{R}}
\newcommand{\Z}{\mathbb{Z}}

\newcommand{\bU}{\mathbf{1}}
\newcommand{\bx}{\mathbf{x}}

\newcommand{\cB}{\mathcal{B}}
\newcommand{\cC}{\mathcal{C}}
\newcommand{\cD}{\mathcal{D}}
\newcommand{\cF}{\mathcal{F}}
\newcommand{\cG}{\mathcal{G}}
\newcommand{\cH}{\mathcal{H}}
\newcommand{\cO}{\mathcal{O}}
\newcommand{\cT}{\mathcal{T}}
\newcommand{\cX}{\mathcal{X}}
\newcommand{\cY}{\mathcal{Y}}

\newcommand{\sA}{\mathscr{A}}
\newcommand{\sB}{\mathscr{B}}
\newcommand{\sC}{\mathscr{C}}
\newcommand{\sD}{\mathscr{D}}
\newcommand{\sE}{\mathscr{E}}
\newcommand{\sF}{\mathscr{F}}
\newcommand{\sG}{\mathscr{G}}
\newcommand{\sH}{\mathscr{H}}
\newcommand{\sL}{\mathscr{L}}
\newcommand{\dO}{\mathscr{O}}
\newcommand{\sS}{\mathscr{S}}
\newcommand{\sT}{\mathscr{T}}
\newcommand{\sX}{\mathscr{X}}
\newcommand{\sY}{\mathscr{Y}}
\newcommand{\sZ}{\mathscr{Z}}

% Debug
\newcommand{\todo}[1]{\begin{color}{blue}{{\bf~[TODO:~#1]}}\end{color}}

% a few handy macros

\renewcommand{\le}{\leqslant}
\renewcommand{\ge}{\geqslant}
\newcommand\matlab{{\sc matlab}}
\newcommand{\goto}{\rightarrow}
\newcommand{\bigo}{{\mathcal O}}
%\newcommand{\half}{\frac{1}{2}}
%\newcommand\implies{\quad\Longrightarrow\quad}
\newcommand\reals{{{\rm l} \kern -.15em {\rm R} }}
\newcommand\complex{{\raisebox{.043ex}{\rule{0.07em}{1.56ex}} \hskip -.35em {\rm C}}}


% macros for matrices/vectors:

% matrix environment for vectors or matrices where elements are centered
\newenvironment{mat}{\left[\begin{array}{ccccccccccccccc}}{\end{array}\right]}
\newcommand\bcm{\begin{mat}}
\newcommand\ecm{\end{mat}}

% matrix environment for vectors or matrices where elements are right justifvied
\newenvironment{rmat}{\left[\begin{array}{rrrrrrrrrrrrr}}{\end{array}\right]}
\newcommand\brm{\begin{rmat}}
\newcommand\erm{\end{rmat}}

% for left brace and a set of choices
%\newenvironment{choices}{\left\{ \begin{array}{ll}}{\end{array}\right.}
\newcommand\when{&\text{if~}}
\newcommand\otherwise{&\text{otherwise}}
% sample usage:
%  \delta_{ij} = \begin{choices} 1 \when i=j, \\ 0 \otherwise \end{choices}


% for labeling and referencing equations:
\newcommand{\eql}{\begin{equation}\label}
\newcommand{\eqn}[1]{(\ref{#1})}
% can then do
%  \eql{eqnlabel}
%  ...
%  \end{equation}
% and refer to it as equation \eqn{eqnlabel}.  


% some useful macros for finite difference methods:
\newcommand\unp{U^{n+1}}
\newcommand\unm{U^{n-1}}

% for chemical reactions:
\newcommand{\react}[1]{\stackrel{K_{#1}}{\rightarrow}}
\newcommand{\reactb}[2]{\stackrel{K_{#1}}{~\stackrel{\rightleftharpoons}
   {\scriptstyle K_{#2}}}~}


\makeatletter
\def\th@plain{%
  \thm@notefont{}% same as heading font
  \itshape % body font
}
\def\th@definition{%
  \thm@notefont{}% same as heading font
  \normalfont % body font
}
\makeatother
\date{}


\title{Lecture-15: Distance from stationarity}


\begin{document}
\maketitle

\section{The convergence theorem}
\begin{thm}[Convergence Theorem] 
Let $X: T \to \sX^N$ be an irreducible and aperiodic Markov chain with transition probability matrix $P$ and stationary distribution $\pi$. 
Then there exist constants $\alpha \in (0,1)$ and $C > 0$ such that
\EQ{
\max_{x \in \sX^N}\norm{\pi_x^t - \pi}_{\text{TV}} \le C\alpha^t.
}
\end{thm}
\begin{proof}
Since the Markov chain $X$ is irreducible and aperiodic, 
there exists a positive integer $r$ such that $P^r$ has strictly positive entries. 
Let $\Pi$ be the matrix with $\abs{\sX}^N$ rows, 
each of which is the row vector $\pi$. 
We chose a $\delta > 0$  sufficiently small such that 
\EQ{
\delta \le \min\set{\frac{P^r(x,y)}{\pi(y)}: x, y \in \sX^N}.
} 
Let $\bar{\delta} = 1-\delta$, then we can define a matrix $Q = \frac{1}{\bar{\delta}}(P^r- \delta\Pi)$. 
We can verify that $Q$ is a stochastic matrix by right multiplying it with vector $\bU$. 
We can also verify that for any stochastic matrix $M$, we have $M\Pi = \Pi$ and if $\pi$ is an invariant distribution of $M$, then $\Pi M = \Pi$. 
We next show by induction that 
\EQ{
P^{rk} = (\delta\Pi + \bar{\delta}Q)^k = (1-\bar{\delta}^k)\Pi+\bar{\delta}^kQ^k.
}
The base case of $k=1$ is true by definition. 
We assume the inductive hypothesis holds true for $k =n$,  then
\EQ{
P^{r(n+1)} = P^{rn}P^r = [(1-\bar{\delta}^n)\Pi+\bar{\delta}^nQ^n]P^r = (1-\bar{\delta}^n)\Pi + \bar{\delta}^nQ^n((1- \bar{\delta})\Pi + \bar{\delta}Q). 
}
The first equality follows from the fact that $\Pi P^r = \Pi$ and the second equality from the definition of stochastic matrix $Q$. 
Since $Q^n\Pi = \Pi$, we have 
\EQ{
P^{r(n+1)} = (1-\bar{\delta}^n)\Pi + \bar{\delta}^nQ^n((1- \bar{\delta})\Pi + \bar{\delta}Q) = (1-\bar{\delta}^{n+1})\Pi +  \bar{\delta}^{n+1}Q^{n+1}. 
}
Hence, the result follows from the induction. 
By post-multiplication with $P^j$, we get 
\EQ{
P^{rk+j} - \Pi = \bar{\delta}^k(Q^kP^j- \Pi).
}
We can write the total-variation distance between $\pi^{t}_x$ and $\pi$ for $t=rk+j$
\EQ{
\norm{\pi^{t}_x-\pi}_{\text{TV}} = \norm{P^{rk+j}(x, \cdot)-\pi(\cdot)}_{\text{TV}} 
%= \frac{1}{2}\sum_{y \in \sX^N}\abs{P^{rk+j}(x, y)-\pi(y)} 
\le \bar{\delta}^k \norm{Q^{k}P^j-\pi}_{\text{TV}} \le  \bar{\delta}^k \le C\alpha^t,
}
for $\alpha = \bar{\delta}^{1/r}$ and $C = 1/\bar{\delta}$. 
\end{proof}
\subsection{Maximal distance from stationarity}
\begin{defn}
The maximal distance between $t$-step distribution $\pi^t$ and stationary distribution $\pi$ over all initial configurations $x \in\sX^N$ is defined as 
\EQ{
d(t) \triangleq \max_{x \in \sX^N}\norm{P^t(x, \cdot), \pi(x)}_{\text{TV}}.
}
The maximal distance between $t$-step distributions $P^t(x, \cdot)$ and $P^t(y, \cdot)$ over all initial configurations $x, y \in\sX^N$ is defined as 
\EQ{
\bar{d}(t) \triangleq \max_{x \in \sX^N}\norm{P^t(x, \cdot), \pi(x)}_{\text{TV}}.
}
\end{defn}
\begin{lem} The following relation between maximal distances is true, 
\EQ{
d(t) \le \bar{d}(t) \le 2d(t). 
}
\end{lem}
\begin{proof} 
It is immediate from the triangle inequality for the total variation distance that $\bar{d}(t) \le 2d(t)$.
To show that $d(t) ? \bar{d}(t)$, note first that since $\pi$ is stationary, 
we have $\pi(A) = ??\sum_{y \in \sX^N}\pi(y)P^t(y, A)$ for any set $A$. 
Therefore, 
\EQ{
P^{t}(x,A)- \pi(A) = \sum_{y \in \sX^N}\pi(y)(P^t(x,A)- P^t(y,A)) \le  \sum_{y \in \sX^N}\pi(y)\norm{P^t(x,\cdot)- P^t(y,\cdot)}_{\text{TV}} \le \bar{d}(t), 
}
by the triangle inequality and the definition of total variation. 
Maximizing the left-hand side over $x$ and $A$ yields $d(t) \le \bar{d}(t)$.  
\end{proof}
\begin{exerc}
Show the following. 
\meq{2}{
&d(t) = \sup_{\mu \in \sM(\sX^N)}\norm{\mu P^t - \pi}_{\text{TV}}, && \sup_{\mu, \nu \in \sM(\sX^N)}\norm{\mu P^t - \nu P^t}_{\text{TV}}.
} 
\end{exerc}
\begin{lem}
The function $\bar{d}$ is sub-multiplicative. 
That is, $\bar{d}(s + t) \le \bar{d}(s)\bar{d}(t)$. 
\end{lem}
\begin{proof} 
Fix $x, y \in \sX^N$ and let $X_s, Y_s$ denote the configuration of a Markov chain with homogeneous transition probability matrix $P$ starting from initial state $x,y$ respectively. 
Let $(X_s, Y_s)$ be the optimal coupling of $P^s(x, \cdot)$ and $P^s(y, \cdot)$. 
Hence 
\EQ{
\norm{P^s(x, \cdot) - P^s(y, \cdot)}_{\text{TV}} = P\set{X_s \neq Y_s}.
}
We can write 
\EQ{
P^{s+t}(x, w) = \sum_{z \in \sX^N}P\set{X_s = z}P^t(z, w) = \E P^t(X_s, w).
}
Hence, we can write for a set $A$, 
\EQ{
P^{s+t}(x, A) - P^{s+t}(y, A) = \E[P^t(X_s, A) - P^t(Y_s, A)] \le \E[\bar{d}(t)\indicator{X_s \neq Y_s}] = \bar{d}(t)P\set{X_s \neq Y_s}.
}
\end{proof}
\end{document}
